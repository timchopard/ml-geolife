\section{Background}


The GeoLifeCLEF challenge has been running for a number of years. Each year, participants are tasked with predicting species distribution, but the challenge has evolved over time, with new datasets, evaluation metrics, and research questions introduced each year. Here, we provide an overview of the some of the recent submissions to GeoLifeCLEF challenge and summarize the key contributions.

In 2021, the GeoLifeCLEF challenge focused on fine-grained visual categorization using remote sensing data. The winning submission by \cite{Seneviratne:CLEF-2021} leveraged contrastive learning to improve species distribution modeling (SDM) from remote sensing imagery. The authors explored the effectiveness of using only RGB imagery and the impact of adding altitude imagery to the model's performance. They introduced a new consistency-based model selection metric to enhance the model's generalization capabilities. The paper outlined potential areas for further research, including the impact of transformations and the utility of the consistency metric.

In 2022, the GeoLifeCLEF challenge shifted its focus to predicting species distribution across the U.S. and France using remote sensing data and other covariates. The second-place submission by \cite{KellenbergerDevis:CLEF2022} proposed a classification approach with a spatial block-label swap regularization during training and an ensemble of deep learning models. Their method achieved a top-30 accuracy of 31.22\% on the private test set, securing second place in the competition. The authors reflected on the results and suggested potential improvements and the importance of species distribution modeling for ecological research.

In 2023, the GeoLifeCLEF challenge introduced a new dataset with single positive labels for each location, making multi-label prediction challenging. The winning submission by \cite{UngKojimaWada:CLEF2023} proposed a three-step training strategy to leverage the single positive labels effectively. The authors introduced several CNN-based models and demonstrated their effectiveness compared to a simple baseline. The paper discussed the challenges of the new dataset and the proposed models' performance, providing detailed results and comparisons.
