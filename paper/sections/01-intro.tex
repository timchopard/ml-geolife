\section{Introduction}

In this research study, we aim to develop a model for predicting plant species in a specific location and time using various environmental factors as predictors. These predictors include satellite images, climatic time series, and other rasterized environmental data such as land cover, human footprint, bioclimatic variables, and soil characteristics. Our motivation behind this challenge is the potential usefulness of accurate plant species prediction in various scenarios related to biodiversity management and conservation, species identification and inventory tools, and education.

We utilized a large-scale training dataset of approximately 5 million plant occurrences in Europe, as well as validation and test sets with over 5,000 and 20,000 plots, respectively. The predicted output will be multi-label, presence-absence data for all present species at each plot. The data covered over 10,000 different plant species, which created significant challendges associated with this task, including learning from single positive labels, dealing with strong class imbalance, multi-modal learning, and handling large-scale datasets.

The potential applications of accurate plant species prediction are numerous. High-resolution maps of species composition and related biodiversity indicators can be created to aid in scientific ecology studies and conservation efforts. The accuracy of species identification tools can be improved by reducing the list of candidate species observable at a given site. Additionally, location-based recommendation services and educational applications with features such as quests or contextualized educational pathways can be developed to facilitate biodiversity inventories and promote environmental education. We believe that our research will contribute to the advancement of plant species prediction and its practical applications in various fields.

% mention that the research is from a kaggle competition
The research was conducted as part of the GeoLifeCLEF 2024 competition on Kaggle \cite{geolifeclef-2024}, which is a part of the LifeCLEF initiative. The competition aims to develop models for predicting plant species in a specific location and time using various environmental factors as predictors.
