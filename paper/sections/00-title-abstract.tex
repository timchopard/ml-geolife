
%%
%% Rights management information.
%% CC-BY is default license.
\copyrightyear{2024}
\copyrightclause{Copyright for this paper by its authors.
  Use permitted under Creative Commons License Attribution 4.0
  International (CC BY 4.0).}

%%
%% This command is for the conference information            %%%%%%% TODO !!! UPDATE THIS %%%%%%
\conference{Woodstock'22: Symposium on the irreproducible science,
  June 07--11, 2022, Woodstock, NY}

%%
%% The "title" command
\title{A better way to format your document for CEUR-WS}

% Authors
\author[1]{Darren Rawlings}[%
email=abc@def.ghi,
url=https://startung.github.io/,
]
\address[1]{University of Groningen, Broerstraat 5, 9712 CP Groningen, Netherlands}

\author[]{Tim Chopard}[%
email=timchopard@pm.me,
url=http://cloudberries.io,
]

\fntext[1]{These authors contributed equally.}

%%
%% The abstract is a short summary of the work to be presented in the
%% article.
\begin{abstract}
  A clear and well-documented \LaTeX{} document is presented as an
  article formatted for publication by CEUR-WS in a conference
  proceedings. Based on the ``ceurart'' document class, this article
  presents and explains many of the common variations, as well as many
  of the formatting elements an author may use in the preparation of
  the documentation of their work.
\end{abstract}

%%
%% Keywords. The author(s) should pick words that accurately describe
%% the work being presented. Separate the keywords with commas.
\begin{keywords}
  Multi-Label classification\sep
  Principal component analysis\sep
  ResNet\sep
  Vision Transformers\sep
  XGBoost\sep
  GeoLifeCLEF 2024\sep
  CEUR-WS
\end{keywords}

%%
%% This command processes the author and affiliation and title
%% information and builds the first part of the formatted document.
\maketitle