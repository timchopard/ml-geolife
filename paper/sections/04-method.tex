\clearpage  % TODO comment this out when finished

\section{Method}

In order to manage the multilabel classification required for this research we implemented an ensemble approach.
This section details the individual architectures and the methods used to combine them.

\subsection{Architectures}

The core architectures used in this project were 18 layer ResNet, XGBoost and [TRANSFORMER]. The outputs of these were
then weighted and combined before the maximum arguments were selected.

\subsubsection{Principal Component Analysis}
Principal Component Analysis (PCA) is a method for reducing dimensionality~\cite{pcapaper}. It functions by projecting the high dimensional data onto the direction of maximum variance, thus retaining key features and reducing noise.



\subsubsection{XGBoost}
XGBoost is an open source gradient tree boosting package~\cite{xgboost}.
For this research we used the xgbregression model It has shown broad success accross a range of tasks, performing on par with or better than most equivalent and Automated Machine learning approaches~\cite{xgbcomp}.
\subsubsection{ResNet}
ResNet (Residual Network) is an architecture for deep neural networks that speeds up the training process through the use of residual connections~\cite{resnet}.
For the purposes of this research we focused on ResNet18, the 18 layer deep variation with some modifications to accomodate the shape of the input data and the required output.
\subsubsection{Transformer}
Vision transformers (ViTs) have emerged as a novel approach to image classification, often outperforming traditional convolutional neural networks (CNNs) by leveraging self-attention mechanisms originally developed for machine translation \cite{vaswani2017attention}. Transformer now underpin some of the most powerful Large Language models \cite{brown2020language}. By tokenizing images into fixed-size patches and encoding them into embeddings for transformer layers to process, ViTs excel at capturing long-range dependencies within an image, enabling a more holistic interpretation of visual context compared to CNN's local focus \cite{dosovitskiy2021image}. This methodological shift allows the models to effectively decipher complex scenes and interactions that are typically challenging for traditional architectures.

Often ViT-based models are often developed for large high resolution imagery. This was not the case for this application where the satellite images were a modest 128x128 pixels. We believed that the ability of transformer based models to capture to intricate regional relationships. Would be beneficial to the task of species classification.

\subsection{Process}

\subsubsection{Preprocessing}


\begin{figure}\label{fig:preproc}
    \begin{center}
        \tikzstyle{databox} = [parentbox, minimum width=40mm, minimum height=12mm, text centered, draw=red!70, fill=red!10, text=fontcolor!70]
\tikzstyle{pcabox} = [parentbox, minimum width=20mm, minimum height=7mm, text centered, draw=customorange!70, fill=customorange!10, text=fontcolor!70]
\tikzstyle{sarrowxgb} = [parentarrow, draw=customorange!70]
\tikzstyle{outbox} = [parentbox, minimum width=40mm, minimum height=12mm, text centered, draw=blue!70, fill=blue!10, text=fontcolor!70]
\tikzstyle{sarrowout} = [parentarrow, draw=blue!50]


\begin{tikzpicture}

    
% Data
\node (bioclim) [databox] {Bioclimatic Rasters};
% \node (landsat) [databox, below of=bioclim, yshift=-7mm] {LandSat};
\node (metadata) [databox, below of=bioclim, yshift=-7mm] {Metadata};
% \node (sentinel) [databox, below of=metadata, yshift=-7mm] {Sentinel};
    

% PCAs
\node (pca0) [pcabox, right of=bioclim, xshift=30mm, yshift=15mm] {PCA$_0$}; 
\node (pca1) [pcabox, below of=pca0] {PCA$_1$}; 
\node (vertellipsis0) [below of=pca1, yshift=3mm] {\Large{\textcolor{fontcolor!70}{$\cdot$}}};
\node (vertellipsis1) [below of=vertellipsis0, yshift=9mm] {\Large{\textcolor{fontcolor!70}{$\cdot$}}};
\node (vertellipsis2) [below of=vertellipsis1, yshift=9mm] {\Large{\textcolor{fontcolor!70}{$\cdot$}}};
\node (pca2) [pcabox, below of=vertellipsis2, yshift=3mm] {PCA$_n$}; 
\draw[sarrowxgb] (bioclim.east) .. controls +(right:5mm) and +(left:8mm) .. (pca0.west);
\draw[sarrowxgb] (bioclim.east) .. controls +(right:5mm) and +(left:8mm) .. (pca1.west);
\draw[sarrowxgb] (bioclim.east) .. controls +(right:5mm) and +(left:8mm) .. (pca2.west);

\node (output) [outbox, right of=pca1, xshift=40mm, yshift=-15mm] {Processed Data};

\draw[sarrowout] (pca0.east) .. controls +(right:5mm) and +(left:8mm) .. (output.west);
\draw[sarrowout] (pca1.east) .. controls +(right:5mm) and +(left:8mm) .. (output.west);
\draw[sarrowout] (pca2.east) .. controls +(right:5mm) and +(left:8mm) .. (output.west);
\draw[sarrowout] (metadata.east) .. controls +(right:60mm) and +(left:10mm) .. (output.west);


% \node (baseens) [baseens, right of=bioclim, yshift=-43mm, xshift=40mm] {Score Ensemble};
% \draw[sarrowxgb] (bioclim.east) .. controls +(right:5mm) and +(left:8mm) .. (baseens.west);
% \draw[sarrowxgb] (landsat.east) .. controls +(right:5mm) and +(left:8mm) .. (baseens.west);
% \draw[sarrowxgb] (metadata.east) .. controls +(right:5mm) and +(left:8mm) .. (baseens.west);
% \draw[sarrowxgb] (sentinel.east) .. controls +(right:5mm) and +(left:8mm) .. (baseens.west);



\end{tikzpicture} 
        \caption{Preprocessing the data for use in the count prediction and XGBoost pseudo-probability models}
    \end{center}
\end{figure}



\subsubsection{Main Process}
\newcommand{\nspecies}{N_{\text{species}}}

There were two key steps to the process of species selection. First the core models which generated pseudo-probabilities and were then weighted and combined.
Second separeate models were used to determine the expected number $(\nspecies)$ of species per survey.
These steps came together with the expected counts being used to select the top  $\nspecies$ psuedo-probabilities per survey.

% \subsubsection{Species scoring}
% \begin{figure}
%     \begin{center}
%         % Tikz shape definitions
\tikzstyle{nnbox} = [rectangle, rounded corners=0.5mm, minimum width=35mm, minimum height=10mm, text centered, draw=customteal, fill=customteal!30, text=fontcolor]
\tikzstyle{transformerbox} = [rectangle, rounded corners=0.5mm, minimum width=35mm, minimum height=10mm, text centered, draw=customgreen, fill=customgreen!30, text=fontcolor]
\tikzstyle{xgbbox} = [rectangle, rounded corners=0.5mm, minimum width=35mm, minimum height=10mm, text centered, draw=customorange, fill=customorange!30, text=fontcolor]
\tikzstyle{pcabox} = [rectangle, rounded corners=2mm, minimum width=20mm, minimum height=7mm, text centered, draw=customorange!70, fill=customorange!10, text=fontcolor!70]
\tikzstyle{databox} = [rectangle, minimum width=40mm, minimum height=12mm, text centered, draw=red!70, fill=red!10, text=fontcolor!70]
\tikzstyle{convnode} = [rectangle, rounded corners=3mm, minimum width=6mm, minimum height=6mm, fill=customteal!40, draw=customteal!80]
\tikzstyle{output} = [rectangle, minimum width=30mm, minimum height=10mm, fill=blue!20, draw=blue!50]
\tikzstyle{sarrowxgb} = [->, thick, draw=customorange!70]
\tikzstyle{sarrownn} = [->, thick, draw=customteal!70]
\tikzstyle{sarrowtr} = [->, thick, draw=customgreen!70]
\tikzstyle{sarrowout} = [->, thick, draw=blue!50]


\begin{tikzpicture}

% Data
\node (bioclim) [databox] {Bioclimatic Rasters};
\node (landsat) [databox, below of=bioclim, yshift=-7mm] {LandSat};
\node (metadata) [databox, below of=landsat, yshift=-7mm] {Metadata};
\node (sentinel) [databox, below of=metadata, yshift=-7mm] {Sentinel};

% PCAs
\node (pca0) [pcabox, right of=bioclim, xshift=30mm, yshift=30mm] {PCA$_0$}; 
\node (pca1) [pcabox, below of=pca0] {PCA$_1$}; 
\node (vertellipsis0) [below of=pca1, yshift=3mm] {\Large{\textcolor{fontcolor!70}{$\cdot$}}};
\node (vertellipsis1) [below of=vertellipsis0, yshift=9mm] {\Large{\textcolor{fontcolor!70}{$\cdot$}}};
\node (vertellipsis2) [below of=vertellipsis1, yshift=9mm] {\Large{\textcolor{fontcolor!70}{$\cdot$}}};
\node (pca2) [pcabox, below of=vertellipsis2, yshift=3mm] {PCA$_n$}; 
\draw[sarrowxgb] (bioclim.east) .. controls +(right:5mm) and +(left:8mm) .. (pca0.west);
\draw[sarrowxgb] (bioclim.east) .. controls +(right:5mm) and +(left:8mm) .. (pca1.west);
\draw[sarrowxgb] (bioclim.east) .. controls +(right:5mm) and +(left:8mm) .. (pca2.west);

% XGB
\node (xgb) [xgbbox, right of=pca0, xshift=30mm, yshift=-15mm] {XGBoostRegressor};
\draw[sarrowxgb] (pca0.east) .. controls +(right:5mm) and +(left:8mm) .. (xgb.west);
\draw[sarrowxgb] (pca1.east) .. controls +(right:5mm) and +(left:8mm) .. (xgb.west);
\draw[sarrowxgb] (pca2.east) .. controls +(right:5mm) and +(left:8mm) .. (xgb.west);
\draw[sarrowxgb] (landsat.east) .. controls +(right:35mm) and +(left:10mm) .. (xgb.west);
\draw[sarrowxgb] (metadata.east) .. controls +(right:35mm) and +(left:10mm) .. (xgb.west);

% ResNet
\node (nn0) [nnbox, below of=xgb, yshift=-15mm] {ResNet18};
\node (nn1) [nnbox, below of=nn0, yshift=-3mm] {ResNet18};
\node (nn2) [nnbox, below of=nn1, yshift=-3mm] {ResNet18};
\node (conv) [convnode, right of=nn1, xshift=20mm] {};
\draw[sarrownn] (bioclim.east) .. controls +(right:25mm) and +(left:20mm) .. (nn0.west);
\draw[sarrownn] (landsat.east) .. controls +(right:25mm) and +(left:20mm) .. (nn1.west);
\draw[sarrownn] (sentinel.east) .. controls +(right:25mm) and +(left:20mm) .. (nn2.west);
\draw[sarrownn] (nn0.east) .. controls +(right:7mm) and +(left:7mm) .. (conv.west);
\draw[sarrownn] (nn1.east) .. controls +(right:7mm) and +(left:7mm) .. (conv.west);
\draw[sarrownn] (nn2.east) .. controls +(right:7mm) and +(left:7mm) .. (conv.west);

% Transformer
\node (transformer) [transformerbox, below of=nn2, yshift=-15mm] {Transformer};
\draw[sarrowtr] (bioclim.east) .. controls +(right:25mm) and +(left:20mm) .. (transformer.west);
\draw[sarrowtr] (landsat.east) .. controls +(right:25mm) and +(left:20mm) .. (transformer.west);
\draw[sarrowtr] (metadata.east) .. controls +(right:25mm) and +(left:20mm) .. (transformer.west);
\draw[sarrowtr] (sentinel.east) .. controls +(right:25mm) and +(left:20mm) .. (transformer.west);

% Output
\node (output) [output, right of=transformer, xshift=30mm, yshift=-15mm] {Output};
\draw[sarrowout] (xgb.east) .. controls +(right:25mm) and +(north:50mm) .. (output.north);
\draw[sarrowout] (conv.east) .. controls +(right:5mm) and +(north:20mm) .. (output.north);
\draw[sarrowout] (transformer.east) .. controls +(right:15mm) and +(north:10mm) .. (output.north);

\end{tikzpicture}
%         \caption{The structure of the scoring ensemble from data through to individual species score outputs.}
%     \end{center}\label{fig:score-structure}
% \end{figure}
% \subsubsection{Species counts}
% \begin{figure}
%     \begin{center}
%         % Tikz shape definitions
\tikzstyle{parentbox} = [rectangle, rounded corners=1mm] % TODO
\tikzstyle{nnbox} = [parentbox, minimum width=35mm, minimum height=10mm, text centered, draw=customteal, fill=customteal!30, text=fontcolor]
\tikzstyle{transformerbox} = [parentbox, minimum width=35mm, minimum height=10mm, text centered, draw=customgreen, fill=customgreen!30, text=fontcolor]
\tikzstyle{xgbbox} = [parentbox, minimum width=35mm, minimum height=10mm, text centered, draw=customorange, fill=customorange!30, text=fontcolor]
\tikzstyle{pcabox} = [parentbox, minimum width=20mm, minimum height=7mm, text centered, draw=customorange!70, fill=customorange!10, text=fontcolor!70]
\tikzstyle{databox} = [parentbox, minimum width=40mm, minimum height=12mm, text centered, draw=red!70, fill=red!10, text=fontcolor!70]
\tikzstyle{convnode} = [rectangle, rounded corners=3mm, minimum width=6mm, minimum height=6mm, fill=customteal!40, draw=customteal!80]
\tikzstyle{output} = [parentbox, minimum width=30mm, minimum height=10mm, fill=blue!20, draw=blue!50]

\tikzstyle{baseens} = [parentbox, minimum height=30mm, minimum height=20mm, fill=customorange!30, draw=customorange!70]

% Arrows
\tikzstyle{parentarrow} = [-stealth, line width=0.5mm]
\tikzstyle{sarrowxgb} = [parentarrow, draw=customorange!70]
\tikzstyle{sarrownn} = [parentarrow, draw=customteal!70]
\tikzstyle{sarrowtr} = [parentarrow, draw=customgreen!70]
\tikzstyle{sarrowout} = [parentarrow, draw=blue!50]


\begin{tikzpicture}

    
% Data
\node (bioclim) [databox] {Bioclimatic Rasters};
\node (landsat) [databox, below of=bioclim, yshift=-7mm] {LandSat};
\node (metadata) [databox, below of=landsat, yshift=-7mm] {Metadata};
\node (sentinel) [databox, below of=metadata, yshift=-7mm] {Sentinel};
    
\node (baseens) [baseens, right of=bioclim, yshift=-43mm, xshift=40mm] {Score Ensemble};
\draw[sarrowxgb] (bioclim.east) .. controls +(right:5mm) and +(left:8mm) .. (baseens.west);
\draw[sarrowxgb] (landsat.east) .. controls +(right:5mm) and +(left:8mm) .. (baseens.west);
\draw[sarrowxgb] (metadata.east) .. controls +(right:5mm) and +(left:8mm) .. (baseens.west);
\draw[sarrowxgb] (sentinel.east) .. controls +(right:5mm) and +(left:8mm) .. (baseens.west);


    % % PCAs
% \node (pca0) [pcabox, right of=bioclim, xshift=30mm, yshift=30mm] {PCA$_0$}; 
% \node (pca1) [pcabox, below of=pca0] {PCA$_1$}; 
% \node (vertellipsis0) [below of=pca1, yshift=3mm] {\Large{\textcolor{fontcolor!70}{$\cdot$}}};
% \node (vertellipsis1) [below of=vertellipsis0, yshift=9mm] {\Large{\textcolor{fontcolor!70}{$\cdot$}}};
% \node (vertellipsis2) [below of=vertellipsis1, yshift=9mm] {\Large{\textcolor{fontcolor!70}{$\cdot$}}};
% \node (pca2) [pcabox, below of=vertellipsis2, yshift=3mm] {PCA$_n$}; 
% \draw[sarrowxgb] (bioclim.east) .. controls +(right:5mm) and +(left:8mm) .. (pca0.west);
% \draw[sarrowxgb] (bioclim.east) .. controls +(right:5mm) and +(left:8mm) .. (pca1.west);
% \draw[sarrowxgb] (bioclim.east) .. controls +(right:5mm) and +(left:8mm) .. (pca2.west);

% % XGB
% \node (xgb) [xgbbox, right of=pca0, xshift=30mm, yshift=-15mm] {XGBoostRegressor};
% \draw[sarrowxgb] (pca0.east) .. controls +(right:5mm) and +(left:8mm) .. (xgb.west);
% \draw[sarrowxgb] (pca1.east) .. controls +(right:5mm) and +(left:8mm) .. (xgb.west);
% \draw[sarrowxgb] (pca2.east) .. controls +(right:5mm) and +(left:8mm) .. (xgb.west);
% \draw[sarrowxgb] (landsat.east) .. controls +(right:35mm) and +(left:10mm) .. (xgb.west);
% \draw[sarrowxgb] (metadata.east) .. controls +(right:35mm) and +(left:10mm) .. (xgb.west);

% % ResNet
% \node (nn0) [nnbox, below of=xgb, yshift=-15mm] {ResNet18};
% \node (nn1) [nnbox, below of=nn0, yshift=-3mm] {ResNet18};
% \node (nn2) [nnbox, below of=nn1, yshift=-3mm] {ResNet18};
% \node (conv) [convnode, right of=nn1, xshift=20mm] {};
% \draw[sarrownn] (bioclim.east) .. controls +(right:25mm) and +(left:20mm) .. (nn0.west);
% \draw[sarrownn] (landsat.east) .. controls +(right:25mm) and +(left:20mm) .. (nn1.west);
% \draw[sarrownn] (sentinel.east) .. controls +(right:25mm) and +(left:20mm) .. (nn2.west);
% \draw[sarrownn] (nn0.east) .. controls +(right:7mm) and +(left:7mm) .. (conv.west);
% \draw[sarrownn] (nn1.east) .. controls +(right:7mm) and +(left:7mm) .. (conv.west);
% \draw[sarrownn] (nn2.east) .. controls +(right:7mm) and +(left:7mm) .. (conv.west);

% % Transformer
% \node (transformer) [transformerbox, below of=nn2, yshift=-15mm] {Transformer};
% \draw[sarrowtr] (bioclim.east) .. controls +(right:25mm) and +(left:20mm) .. (transformer.west);
% \draw[sarrowtr] (landsat.east) .. controls +(right:25mm) and +(left:20mm) .. (transformer.west);
% \draw[sarrowtr] (metadata.east) .. controls +(right:25mm) and +(left:20mm) .. (transformer.west);
% \draw[sarrowtr] (sentinel.east) .. controls +(right:25mm) and +(left:20mm) .. (transformer.west);

% % Output
% \node (output) [output, right of=transformer, xshift=30mm, yshift=-15mm] {Output};
% \draw[sarrowout] (xgb.east) .. controls +(right:25mm) and +(north:50mm) .. (output.north);
% \draw[sarrowout] (conv.east) .. controls +(right:5mm) and +(north:20mm) .. (output.north);
% \draw[sarrowout] (transformer.east) .. controls +(right:15mm) and +(north:10mm) .. (output.north);

\end{tikzpicture} 
%         \caption{The counts ensemble through to final outputs.}
%     \end{center}\label{fig:count-structure}
% \end{figure}
% \subsection{Metrics}

\newcommand{\fp}{\text{FP}}
\newcommand{\fn}{\text{FN}}
\newcommand{\tp}{\text{FP}}

The main scoring metric used was the micro-averaged F1 score as shown in equation \ref{eq:mf1} which is calculated from the precision $\frac{\tp}{\tp + \fp}$ and the recall $\frac{\tp}{\tp + \fn}$ for each individual class $i$.
Where $\tp$ is the true positives, $\fp$ is the false positives and $\fn$ is the false negatives.

\begin{equation}\label{eq:mf1}
    \text{F1}_{micro} = \frac{1}{N}\sum_{i=1}^N\frac{2 \cdot \tp_i}{2 \cdot \tp_i + \fp_i + \fn_i}
\end{equation}
